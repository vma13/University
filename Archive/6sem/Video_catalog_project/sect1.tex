\section{MVC}

\textbf{Model-view-controller} (MVC, «Модель - Представление - Контроллер») -- шаблон проектирования, в котором модель данных приложения, пользовательский интерфейс и управляющая логика разделены на три отдельных компонента так, что модификация одного из компонентов оказывает минимальное воздействие на остальные. Шаблон MVC позволяет разделить данные, представление и обработку действий пользователя на три отдельных компонента:

\begin{itemize}
\item Модель (Model). Модель предоставляет данные (обычно для View), а также реагирует на запросы (обычно от контроллера), изменяя своё состояние.
\item Представление (View). Отвечает за отображение информации (пользовательский интерфейс).
\item Поведение (Controller). Интерпретирует данные, введённые пользователем, и информирует модель и представление о необходимости соответствующей реакции.
\end{itemize}

Важно отметить, что как представление, так и поведение зависят от модели. Однако модель не зависит ни от представления, ни от поведения. Это одно из ключевых достоинств подобного разделения. Оно позволяет строить модель независимо от визуального представления, а также создавать несколько различных представлений для одной модели.

\endinput



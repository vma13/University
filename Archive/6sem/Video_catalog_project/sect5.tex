\section{Приложение}
\vspace*{1cm}
\subsection{Пример работы}
\vspace*{1cm}
Приведем пример работы тех изменений, которые мы внесли в эталонный проект.\\
\begin{figure}[h!]
\begin{center}
\includegraphics[scale=0.35]{image/1.png}
\end{center}
\newpage
\caption{Домашняя страничка}
\end{figure}
\newpage
Далее рассматривается, как будет выглядеть интерфейс просмотра альбома у конкретной персоны и фильма.\\
\begin{figure}[h!]
\begin{center}
\includegraphics[scale=0.25]{image/2.png}
\end{center}
\caption{Альбом у персоны}
\end{figure}
\begin{figure}[h!]
\begin{center}
\includegraphics[scale=0.25]{image/3.png}
\end{center}
\caption{Альбом у фильма}
\end{figure}
\newpage
Для того чтобы добавить новый альбом, достаточно перейти по ссылке "Новый альбом". При нажатии произойдет перенаправление на страничку создания альбома.\\
\begin{figure}[h!]
\begin{center}
\includegraphics[scale=0.35]{image/4.png}
\end{center}
\caption{Добавление Альбома}
\end{figure}
\newpage
Если просто нажать кнопку "Сохранить" то отобразится сообщение об ошибке.\\
\begin{figure}[h!]
\begin{center}
\includegraphics[scale=0.35]{image/5.png}
\end{center}
\caption{Ошибка}
\end{figure}
Альбом должен быть закреплен, либо за персоной, либо за фильмом и никак иначе. Также продемонстрирован наглядный пример работы валидатора: название альбома должно существовать и должно содержать от 3 до 40 символов в названии
\newpage
\endinput

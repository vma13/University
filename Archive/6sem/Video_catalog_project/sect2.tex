\section{Структура базового проекта}
Перед тем как приступать к рассмотрению индивидуального задания рассмотрим структуру базового проекта. Для этого воспользуемся диаграммой классов UML. Диаграммы классов используются при моделировании программных систем (ПС) наиболее часто. Они являются одной из форм статического описания системы с точки зрения ее проектирования, показывая ее структуру. Диаграмма классов не отображает динамическое поведение объектов изображенных на ней классов. На диаграмме классов показываются классы, интерфейсы и отношения между ними.

\begin{figure}[h!]
\begin{center}
\includegraphics[scale=0.6]{image/class_main.png}
\end{center}
\caption{Диаграмма классов эталонного проекта}
\end{figure}

Из данной диаграммы видно, что в системе присутствую такие классы как:

\begin{itemize}
\item Фильм
\item Жанр
\item Страна
\item Персона
\item Пользователь
\end{itemize}

Класс \textbf{Фильм} -- класс хранилище, которое содержит информацию о фильмах: название, слоган, режиссер, год выпуска и т.д. Он находится в отношении один ко многим с классами \textbf{Жанр} и \textbf{Страна}, а классом \textbf{Персона} в отношении многие ко многим. Класс \textbf{Пользователь} не связан отношениями с другими классами, он содержит информацию о пользователях системы: имя, логин, адрес электронной почты, пол, дату рождения и пароль.

\endinput






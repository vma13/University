\section{Введение}
Последние пять лет ознаменовались фантастическим развитием Интернета и новых способов общения между людьми. На переднем крае этого явления находится \textit{World Wide Web (WWW)}. Ежедневно в этой новой коммуникационной среде открываются тысячи новых сайтов, а потребителям предлагаются новые виды услуг. Вместе с бурным развитием рынка появился огромный спрос на новые технологии и разработчиков, владеющих ими. Комплексная веб разработка сайтов различной тематики и направленности предусматривает создание нового или оптимизацию под нужные характеристики уже готового шаблона сайта, выбор и установку наиболее подходящей системы управления контентом и, при необходимости, заполнение ресурса контентом.Чтобы создать удобный и функциональный web-сайт используют различные технические средства, например HTML,  JavaSvript, Flash, различные СУБД. В данной работе были использованы HTML, СУБД PostgreSQL и платформа \textit{RubyOnRails}.\\

\hspace*{0.5cm}\textit{RubyOnRails} -- это полноценный, многоуровневый Фреймворк для построения веб-приложений, использующих базы данных, который основан на архитектуре Модель-Представление-Контроллер (Model-View-Controller, MVC). Динамичный \textit{AJAX} -- интерфейс, обработка запросов и выдача данных в контроллерах, предметная область, отраженная в базе данных, — для всего этого Rails предоставляет однородную среду разработки на Ruby.\\

\endinput


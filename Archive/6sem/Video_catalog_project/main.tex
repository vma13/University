\documentclass[a4paper,12pt,openany]{memoir}
\usepackage[utf8]{inputenc}
\usepackage[russian]{babel}
\usepackage{indentfirst,amsmath,graphicx,pgf}

% Настройка: размер текста
\settypeblocksize{250mm}{160mm}{*}
\setulmargins{*}{*}{1}
\setlrmargins{*}{*}{1}
\checkandfixthelayout

% В русскоязычных книгах и статьях знаки равенства/неравенства, сложения,
% вычитания и умножения принято в случае переноса формулы на другую строку
% дублировать
\renewcommand\ne{\mathchar"3236\mathchar"303D\nobreak
      \discretionary{}{\usefont
      {OMS}{cmsy}{m}{n}\char"36\usefont
      {OT1}{cmr}{m}{n}\char"3D}{}}
\begingroup
\catcode`\+\active\gdef+{\mathchar8235\nobreak\discretionary{}%
 {\usefont{OT1}{cmr}{m}{n}\char43}{}}
\catcode`\-\active\gdef-{\mathchar8704\nobreak\discretionary{}%
 {\usefont{OMS}{cmsy}{m}{n}\char0}{}}
\catcode`\=\active\gdef={\mathchar12349\nobreak\discretionary{}%
 {\usefont{OT1}{cmr}{m}{n}\char61}{}}
\endgroup
\def\cdot{\mathchar8705\nobreak\discretionary{}%
 {\usefont{OMS}{cmsy}{m}{n}\char1}{}}
\def\times{\mathchar8706\nobreak\discretionary{}%
 {\usefont{OMS}{cmsy}{m}{n}\char2}{}}
\def\approx{\mathchar12825\nobreak\discretionary{}%
 {\usefont{OMS}{cmsy}{m}{n}\char25}{}}
\AtBeginDocument{%
\mathcode`\==32768%
\mathcode`\+=32768%
\mathcode`\-=32768%
}

% Для переносов слов с дефисами
\lccode`\-=`\-\defaulthyphenchar=127

% Намного лучше, чем \sloppy
\emergencystretch=2pt
\hfuzz=0.8pt

% Устраняем большие расстояния (по вертикали) в списках
\tightlists

% Для русского языка разделителем в подписях рисунков и таблиц служит точка
\captiondelim{.\space}
% Используем уменьшенный шрифт для подписей
\captionnamefont{\small}
% Величина пробела до и после рисунков и таблиц 
\setlength{\intextsep}{10pt}
% Нумерация рисунков и таблиц сплошная по всему документу
\renewcommand{\thefigure}{\arabic{figure}}
\renewcommand{\thetable}{\arabic{table}}

% Оформление списка литературы
\setbiblabel{[#1]\hfill}
\renewcommand{\bibsection}{%
  \section*{Список литературы и интернет-ресурсов}
  \prebibhook}
\newcommand{\link}[1]{\texttt{#1}}

% Оформление секций
\makeatletter
\renewcommand*{\thesection}{\@arabic\c@section.}
\setsecnumformat{\csname the#1\endcsname\space}
\makeatother

% Оформление страниц -- простейшее (с нумерацией внизу)
\pagestyle{plain}

\begin{document}
\thispagestyle{empty}

\vspace*{-\headheight}\vspace*{-\headsep}

{\centering\textsc{ 
министерство образования и науки РФ\\
московский государственный индустриальный университет\\
кафедра информационных систем и технологий\\
}

\vspace{4cm plus 1mm minus 1mm}

\begin{flushright}
\begin{tabular}{l}
Руководители работы:\\
доцент, к.т.н. Куприянов Д.Ю.\\
ассистент Александров А.И.
\end{tabular}
\end{flushright}

\vspace{3cm plus 1mm minus 1mm}

Войнов Максим Александрович

\vspace{1cm plus 1mm minus 1mm}
{\large\textbf{
<<Разработка Информационной системы учета успеваемости и посещаемости слушателей ФДО МГИУ>>
}}

\vspace{1cm plus 1mm minus 1mm}

Курсовая работа по дисциплине\\
<<Проектирование и разработка корпоративных информационных систем>>\\
4-й курс, 7-й семестр

\vfill

Москва 2011

}

\newpage
\endinput

% Обратная сторона титульного листа
\setcounter{page}{2}

\vspace*{1cm}
\section*{Аннотация}
\begin{quote} 
Курсовая работа посвящена описанию дипломного проекта <<Информационная система учета успеваемости и посещаемости слушателей ФДО МГИУ>>. В данной работе описываются актуальность темы и постановка задачи с описанием её планируемого функционала, обоснование выбора архитектуры планируемой информационной системы и обзор технологий, которые используются для построения аналогичных систем, описание проектирования системы, описание интерфейсов с примерами скриншотов.\\
{\em
}
\end{quote}
\newpage
\tableofcontents*
\endinput

\newpage
\section{Введение}
Последние пять лет ознаменовались фантастическим развитием Интернета и новых способов общения между людьми. На переднем крае этого явления находится \textit{World Wide Web (WWW)}. Ежедневно в этой новой коммуникационной среде открываются тысячи новых сайтов, а потребителям предлагаются новые виды услуг. Вместе с бурным развитием рынка появился огромный спрос на новые технологии и разработчиков, владеющих ими. Комплексная веб разработка сайтов различной тематики и направленности предусматривает создание нового или оптимизацию под нужные характеристики уже готового шаблона сайта, выбор и установку наиболее подходящей системы управления контентом и, при необходимости, заполнение ресурса контентом.Чтобы создать удобный и функциональный web-сайт используют различные технические средства, например HTML,  JavaSvript, Flash, различные СУБД. В данной работе были использованы HTML, СУБД PostgreSQL и платформа \textit{RubyOnRails}.\\

\hspace*{0.5cm}\textit{RubyOnRails} -- это полноценный, многоуровневый Фреймворк для построения веб-приложений, использующих базы данных, который основан на архитектуре Модель-Представление-Контроллер (Model-View-Controller, MVC). Динамичный \textit{AJAX} -- интерфейс, обработка запросов и выдача данных в контроллерах, предметная область, отраженная в базе данных, — для всего этого Rails предоставляет однородную среду разработки на Ruby.\\

\endinput


\newpage
\section{MVC}

\textbf{Model-view-controller} (MVC, «Модель - Представление - Контроллер») -- шаблон проектирования, в котором модель данных приложения, пользовательский интерфейс и управляющая логика разделены на три отдельных компонента так, что модификация одного из компонентов оказывает минимальное воздействие на остальные. Шаблон MVC позволяет разделить данные, представление и обработку действий пользователя на три отдельных компонента:

\begin{itemize}
\item Модель (Model). Модель предоставляет данные (обычно для View), а также реагирует на запросы (обычно от контроллера), изменяя своё состояние.
\item Представление (View). Отвечает за отображение информации (пользовательский интерфейс).
\item Поведение (Controller). Интерпретирует данные, введённые пользователем, и информирует модель и представление о необходимости соответствующей реакции.
\end{itemize}

Важно отметить, что как представление, так и поведение зависят от модели. Однако модель не зависит ни от представления, ни от поведения. Это одно из ключевых достоинств подобного разделения. Оно позволяет строить модель независимо от визуального представления, а также создавать несколько различных представлений для одной модели.

\endinput



\newpage
\section{����� �������}
\subsection{���������� ������}
�����������  ��������� ��� ���������, ��������������, ���������� � �������� ��������� ������������ �������: ��������������� ����� (check - in - desk), ���������� ���������� (terminal), ������������ (company). ����� ��������� ��������� �� ������ � �������, ������� ���� ����������� ��������� ������ ����� (������� ����������, ������������� ����������, ������������� ����������) �������������� ��� ��������������� ������ (check-in-desk-glight), ���������� ��������� (flight-terminal) � �������, ������������ � ������(flight-status). ��� ��������� ��������������� ��������� ����������: ���� �������, ���� �������, ���� �� ���������� ������, ���� ������, ���� ���, ���� ������� �� ���� �� � �.�.\\
\subsection{����������}
1.\emph{������.} � ������� ������� ������, � ����������� � ���������� MVC, �� ����� �������������� ������ � ����� ���� ������. ��� ���� �������� ��� ��������� ������-��������� ����� ������� � ������������, �������������������� ���������� ���������� � ��������������� �������� (������, ���������, � �.�.). ���� ������, ����������, � �������� �������������� �������. ������ �����������, ��� ���� �������� ������ ��������� ������. ������ ����������� ������. ����� ��������� ������ ���� �������� � ���� ������������� ������. ���� ������������� �������� ������ ��� @attributes, ������������ ��� ������� ���������� ������(���� ��������� ���������� ������������ �����, ��� ��� �������� �� ��������� ���������������� nil'���). ������ ��� ���� - ������������� ���������, ��������� ����� ��������� �� ���������� ���������� ��������� �� ����������. ���� ����� ��� - ������ ����� ������ �� �������. � ���� ����� ������ �������� � ����������, � �������������.\\
\begin{verbatim}
def intialize(attributes = {})
 @attributes = {
  :id => nil,
  :name => nil,
  :description => nil
 }
 attributes.each do |k, v|
  @attributes[k] = v
 end
end
\end{verbatim}

����� �������, ���������� � ������������� �������� � ������������ ������. � ������� ������ �� ����������� ������� ������ � ���������� ����������� �� ����� ��������� ������ ��������� � ���� ������.\\

������� ����������� �������������� ������� � ����������� ������������ �������. �������� ������� �������������� ��������� ������� create\_table():\\

\begin{verbatim}
def CheckInDesk.create_table(connection)
   begin
     connection.do("
      CREATE TABLE check_in_desks(
      id serial PRIMARY KEY,
      name varchar(16) UNIQUE NOT NULL,
      decription text
      ) WITH OIDS
   ")
   return true
   rescue DBI::ProgrammingErroor => e
   return false
   end
end
\end{verbatim}
��������� �������� � ����������� ���������������� OIDs, ����������� ��������� ������ � ����������� �����.SQL-������ ����������� ����� ��������� DBI � ������ ��������� ����������. � ������ ��������� ����� (������� ��� ���������, ������ � SQL-����������, �������� ��������� � �.�.) ����� ���������� false.\\

������, ����� �� ������������ ����� ����� ������� � ������������, ������� ���������� � ����������� ����������������. � ���������������� ���������� ������ ���� ������������� ��������� �����������:\\

1.������������ ������ ���� ������������ �����\\
2.����������, �������������� � �������� �������\\
3.������������ ������ ���� ����� ��� �����\\
4.�������������� ������ ����� ��� �����\\

����� ���� ������� ��� ����������� � model.rb, � ��������� ����� ������ �� �������������� find\_first(connection, id) � ������� ���� �������� find\_all(connection). ����� ���������� ��� ����������� ������ �� ��������.\\

��������� � ������� ���� ��� ������� ��������� � �� �� ������ ���� ������, ��������� ����� check-in-desk() � ������ Flights. �� ��� ������� ������� ������ �������, ��� ���������� ���������� ������ � ������������ ������� � ����.\\

\begin{verbatim}
def check_in_desk(connection)
    res = []
    query = ["SELECT cd.*, cdf.id AS cdf_id
              FROM check_in_desks cd JOIN check_in_desk_flights cdf
              ON (cdf.check_in-desk_id = cd.id)
              WHERE cdf.flight_id = ?", self[:id]]
    connection.select_all(*query) do |r|
    f = CheckInDesk.new           
    r.column_names.each do |c|
    f[c.to_sym] = r[c]
end
res << f
end
return res
end
\end{verbatim}
\begin{itemize}
\item �� ����� � ������� - ������������� ������;
\item res - ������ ����������� ������ CheckInDesk;
\item query - ������, ������ ��������� ������� ������ ������ SQL-�������. ������������ � �ţ ��������� ���������� �������� "?", ����� ���� ��������������� ������������� � �������� ��������� ��������� res.\\
\item ����� Action::Base.connection.select\_all ���������� ������ � ������� � ������������ �� ����� ���������� - �������� query, "���������" � SQL-������. �������, ��� �������� cid[key] ��� ���������� CheckInDesk �������� ��������� � ��� ���� cid.attributes[key.to\_sym](������ ����� ��������� � ����������� Model � ���������� ��� ���� ��� �������� �������).�.�. ����� ������������� ���� - ��������� ���������, �� �������� �� � �������������� ���� �������� to\_sym.\\ 
\end{itemize}

� ������ Model ����� ���������� ��� ������ table\_name() (��� ������ � ��� ����������), ������������ �������� ������� � ��, ������� ������������� ������ ��� �����.\\
\begin{verbatim}
def table_name()
self.class.table_name()
end
def Model.table_name()
self.to_s.gsub(/(\W)/, '_\1').downcase + 's'
end
\end{verbatim}


����� �������������� ���������� ������� ����� �����-������ ����������� ������ �������������, ����� ���� ����� ���������� � ��������� ���� ������������� ��������� 's' (CheckInDesk ���������� check\_in\_desk). ���� ������� ������������� ��������� ��� ���������� ������, � ��������� ������ ��������� �������������� ����� ������ � ���������.\\

���� ��������� ���� � ������� ����� ��������� ������������ ����� ��������� ��������, ������ ������� � ��������������� ������ ������� ���������, �������������� ������� ��������� �������� ��������� �������������. � ������� ���� ����� ��������� �� ��� �������� �� �������.\\

�� ����� �������� � ������ FlightStatus ����������� �������:
\begin{verbatim}
STATUSES = {0 => '�������',
1 => '�������'
2 => '���'
3 => '�������'}
...
status_id integer NOT NULL CONSTRAINT status\_id\_ck CHECK (status_id IN (0,1,2,3,4))
...
\end{verbatim}

�.�. �� �����������, ��� �������� ������� �� ����� ���� ������� �� ���� �������� ����.\\
���� ��� ���������� ������ FlightStatus, �������� ������������� ��� ������� ����� ����������� � ������� ��������� �����������:\\
\begin{verbatim}
FlightStatus::STATUSES[s[:status_id]]
\end{verbatim}
� ������ Helper ����� ������������� �������������� ������.��������, ���� ������������ ����� ���������� ������ ����� ��� ���������� �����, ������ �������� ������ �� ��������� �������, ��� ������, ��� �� ������, ������������ ������ �� ����������������, ������������ �� ���������������.\\
�����, ������������ ��� ������ �� �� id:
\begin{verbatim}
def convert_to_cid_name(id)
ans = CheckInDesk.find_first(@db, id)
return ans[:name]
end
\end{verbatim}
��� ������ � ���� ������� �� ����� ������������ ����� ��� ������ timestamp. ���� ��� ������������ ������ ����� YYYY.MM.DD HH:MI(�����, � �������...). �� ����� ��������� �������� ���������� �������� �������� ������� ��� ������ ������� Ruby Time.now
2.\emph{����������.}���� HTML ������������ ����� ��������, ��������� � �������������� HTML ����������. ����������� ������ ����� �������� ���� ���������� �� �������������. ����� ���� ��� ������������ �������� ����� � ��������� ������� ţ ���������, ���������� �� �ţ �������� � ���������, ���������� �� �������(�������).\\

������������� ��������� ������������ � ����������� � �������� ������������ ����������� � aero.rb, ������� � �������� ����� ��������. ���������� ������� ����� ����, �������� ����� ������ ��� ������ ���������� � ����������.\\

\begin{verbatim}
DEFAULT_CONTROLLER = 'Flights'
DEFAULT_ACTION = 'departure_list'
               def render()
               cgi = CGI.new('html4')
               begin
                unless cgi.params.include?('controller') or
                valid_controllers.include?(cgi.params['controller'][0])
                 cgi.params['controller'] = [DEFAULT_CONTROLLER]
                 cgi.params['action'] = [DEFAULT_ACTION]
               end
                c = eval(cgi.params['controller'][0] + 'Controller').new(cgi)
                cgi.out({
                 "type" => "text/html; charset=utf-8",
                 "language" => "ru"
                  }){ c.response() }
                      rescue Exception => e
                      cgi.out({
                       "type" => "text/html; charset=utf-8",
                       "language" => "ru"
                        }){ display_errors(e) }
               end
             end
\end{verbatim}
�������� ������� ������� ����������� � ���������:
\begin{itemize}
\item ��������� ��������� CGI ��� HTML 4.0; ��� ������������� ������, ����� ������� ���������� ��������� ������� � ���� ���: key\_value\_set, ��� value\_set - ������ �� ������ ��� ����� ��������.
\item ����� hes\_key?(key) ����������� � ���� � ���������� ���������� �������� true or false, � ����������� �� ����, �������� �� ��� ���� key.
\item ��������� � �������� ����������� ����� Controller, �������� �������� ������-����������� � ������� ��� ���������, ������� cgi � �����������.
\item ����� cgi.out ����� http-��������� � ���� c.response() � �����. response() ��������� � ��������� html-��������� header � footer �� ����� template/layouts.
\end{itemize}
������������� �� ���������� ���������� �������, ����� �������, ��� � ����� ���������� ������ �������� ��������� �����:\\

1.������� render() �������� �������� ���������� � @cgi.params;\\
2.������������ ������� (�����������) ������ Model ��� ��� ���������� ������������ ��������� � �������������� �������;\\
3.������������� ����������� � ������� ������� render\_template().\\

��������, ������������ ����� ���������� ���������� � ��������� �����,  ������������� �������� ���������� � ����������. ����� @cgi.params - ��������� ���������, � �������� @cgi.params[key] - ������� (� ����� ������ - ��������������).\\
\begin{verbatim}
def show()
@item = nil
if @cgi.params.has_key?('id) and @cgi.params['id'][0] != '''
@item = Flight.find_first(@db, @cgi.params['id'][0])
end
render_template(@item ? 'show' : 'not_found')
end
\end{verbatim}

1.���� ������ ��� ������� �������� id, ��� ������ ������ ������ Flight ������� ������ ������ � ����������� ������������ ������������� ������ ���������� @item (������ ���������� ������������� ������ � �����);\\
2.���������� ������������� � ����������� �� ����, ��� �� ������ ������������� ����.\\

@item - ��������� ������ � ���������� ���������� �����������, ����� ������������ � �������������. � ������� ������ Controller ���������� ����� filter\_for\_params(), ������� �������� �� ���� ��������� ���������� ������ ��������� � ��������� ���� item[attribute\_key] � ������� �� ��� �������� item[], �������� ������ ���������� � [], � ���������� ��������������� ���.\\

���������� ����� ���������� ��������������� ������ ��� �����: � ���� ���������� ���������� �������������� ����� � ������. ����� ������� ���������������� �������� ���������� @item(��������� CheckInDeskFlight), ����� ���� ��� ����������� � ��������� ������� ��� ������������ item.save(connection).

\begin{verbatim}
def attach()
params = filter_for_params()
@item = CheckInDeskFlight.new
params.each do |k, v|
@item[k] = v[0] if k != 'id' and v != ''
end
@item.save(@db)
render_template(@action)
end
\end{verbatim}
\begin{itemize}
\item ���������� ������� ���������� ��� params;\\
\item C������ ��������� ���������� ������ CheckInDeskFlight � ��������� ��� ��� @attributes ����������� ����������. ��� ���������� � ������� ������������� �������� ����� ���� � ������������ �� �������� ������ ��� ���������� ���������� (find\_first()), � ����� � ����� ������ �� ����� �������������\\
\item ������ ����������� � ��, ����� ���� ���������� ��������� �������������.\\
\end{itemize}
3.\emph{�������������}
����� ������������� (templates) � ����� � Ruby, �������� ������������ ������ ������.��� ������ - �������� html-���� � ������������� �������� ������� render\_template(), � ������� ����� ���������� ��������� ������������� ����� ������������� ��������.\\
\begin{verbatim}
def render_template(name, mode = :rb)
if mode == :rb
f = File.new("templates
/#{Convertors.class_name_to_controller_dir(@controller)}/#
{name}.rb")
html = eval(f.read)
f.close
return html
else
f = File.new("templates/#{@controller}/#{name}")
html = f.read
f.close
return html
end
end
def Convertors.class_name_to_controller_dir(class_name)
class_name.to_s.gsub(/(\W)/, '_\1').downcase
end
\end{verbatim}
1.���� ������� ���������� � ����� ����������, ��������������, ��� ���� �������� - ��� �����-������������� ��� ����������. ���������� ��� ����������� �������� .rb � ���� �� ����� �� �������� templates ������������ �������� class\_name\_to\_controller\_dir() (��� ������� ���������� ��� ������������� ������� table-name());\\
2.�� ���������������� ����� ����������� ������, ����� ���� ���������������� � �������� �� ���������. 
\vspace*{1��}������ � �������������� ����������� ��������� ��������: �� ����� �������� � ������������� ������ ���������� ������ ������. ��� ������� �� ��� ����� ����������� ��������� html-���� � "��������������" ��� � ��� ���������� �������.\\
\begin{verbatim}
@items.map do |i|
count += 1
"
<tr class = 'list#{count%2}'>
<td>#{i[:name]}</td>
<td>#{i[:description]}</td>
<td>
#{action_links(@controller, i[:id], @user)}
</td>
15
</tr>
"
end.join("\n")
\end{verbatim}
��� �������� ���� �� ����� ����������� � �������������� ���������� ������������ � ��� ������. ��������� ����� ���� ����������, �� ��������� ��������: ��� "��������" ������ � ����� �� ������ ���������� ������������ ������ ����� ������������ �� ������ ������ �����. ��� ����� ������� � ������ Helper ��������� �������, ����������� ����� ������ ����� ������ ���� <select>. ��� ������� ���������� ������ � ������ html-�����. ������� �� ������ ���� � ������������� ������� � � ��������� ���������� �� ���������.\\
\begin{verbatim}
def check_in_desk_select(name, selected, is_nil)
"<select name = '#{name}'>" +
CheckInDesk.find_all(@db).map do |c|
if c[:id].to_i == selected.to_i && selected!=0
"<option value = '#{c[:id]}' selected>#{c[:name]}</option>"
else
"<option value = '#{c[:id]}'>#{c[:name]}</option>"
end
end.join("\n") + (is_nil ?
(selected==0 ?
"<option value = '' selected> </option>":"<option value=''></option>")
: "") + "</select>"
end
\end{verbatim}
1.�� ����� � ������� - ��� ���������, �������� �������� �� ���������� (name), id �������-�������� �� ��������� (selected) � ��������, ������������, ����� �� name ������� ������ �������� (is\_nil);\\
2.��� ������� ������� �� ������� ������� ��� ��� � ������ ��������� ��������; � cgi.params �� ����� ���������� �������������;
3.���� selected == 0 � is\_nil == true, ��������� �� ��������� �������� ������ ������.

\newpage
\section{�������������� �������}
\emph{���������� ������}. ��������� ����������� ������ "���������". ������ ������ ������ ��������� ��������� ������: 
\begin{itemize}
\item ����������� ���������� ��� ������ � ������ �������.\\
\item ����������� ����������� �� ���� - ��� ����������� �� ������ �������� � ����� ������������� ����������� ���������� �����. (������� ������ ����������� ������ � ����������� ����������, ����� �������� � ����� � ���������� � �������� ����� ���� � ����� ����� � ����. �������, ��� �� ���� ��������� 20 ����� �� 6 ���� � ������ ���� (a,b,c,d,e,f) ����� �������� �� �������. ����� ����� ���������, ������������������ ������ ������).\\
\item ����������� ������� �������� ������� ���� ��� �����.\\
\end{itemize}
\subsection{������� ������}

��������� ����� ��������� ticket, �� ���������� ���������� (id, flight\_id, passport). �������� ������� ����� ������������� ��������� �������: 
\begin{verbatim}
 def Ticket.create_table(connection)
    begin
      connection.do("
CREATE TABLE tickets(
  id serial PRIMARY KEY,
  flight_id integer REFERENCES flights(id) NOT NULL,
  passport varchar(20) NOT NULL
) WITH OIDS
        ")

\end{verbatim}
������� ����������� � 2 ������, ������������ ���� �� ������� ���������� ����� � ����� ������� ������: 
\begin{verbatim}
  def flight(connection)
    Flight.find_first(connection, @attributes[:flight_id])
  end

    def coupon(connection)
    Coupon.find_by(connection, 'ticket_id', @attributes[:id])
  end
\end{verbatim}

������� ������������� ��� ������� ���������:
\begin{verbatim}
 <caption>������������ ���������� �� ���� #{@flight[:code]}:</caption>
  <thead>
  <tbody>
    <tr>
      <th>
        �����<br>
        <div style = 'margin-left: 10px;'>�����:</div>
        <div style = 'margin-left: 10px;'>�����:</div>
        <div style = 'margin-left: 10px;'>��������:</div>
      </th>
      <td>
        &nbsp;<br>
        #{@flight[:departure_place]}<br>
        #{@flight[:departure_date]}<br>
        #{@flight[:departure_airport]}
      </td>
    </tr>
    <tr>
      <th>
        �������<br>
        <div style = 'margin-left: 10px;'>�����:</div>
        <div style = 'margin-left: 10px;'>�����:</div>
        <div style = 'margin-left: 10px;'>��������:</div>
      </th>
      <td>
        &nbsp;<br>
        #{@flight[:arrival_place]}<br>
        #{@flight[:arrival_date]}<br>
        #{@flight[:arrival_airport]}
      </td>
    </tr>
    <tr>
      <th>������������:</th>
      <td>#{@flight.company_name()}</td>
    </tr>
    <tr>
      <th>���� ��������:</th>
      <td>#{@flight[:is_departure] ? '����������' : '�����������'}</td>
    </tr>
    <tr>
      <th>���:</th>
      <td>#{@row}</td>
    </tr>
    <tr>
      <th>�����:</th>
      <td>#{@seat.upcase}</td>
    </tr>
    <tr>
      <th>����� ��������:</th>
      <td><input type = 'text' name = 'item[passport]' size = '30'></td>
    </tr>
  </tbody>
  <tfoot>
    <tr>
      <th colspan = '2'>
        <input type = 'submit' value = '���������� ���������'>
        <input type = 'button' value = '�����, � ������ ����'
               onclick = 'javascript:document.location=\"aero.rb?controller=Tickets&action=index&flight_id=#{@flight[:id]}\"'>
      </th>
    </tr>
  </tfoot>
</table>
\end{verbatim}
� ������ ������������� ����������� ���������� ������.

\section{������� ������}


\includegraphics[scale=0.35]{sc1.png}
����� ������ ������ \\
\includegraphics[scale=0.35]{sc2.png}
������ �������\\
\includegraphics[scale=0.35]{sc3.png}
��������������� �����\\
\includegraphics[scale=0.35]{sc4.png}
������ ���������� ������\\

\section{Реализация}
\subsection{Модель}
Для начала определим, какие классы необходимо добавить в систему.  Для хранения информации о альбомах добавим новый класс \textbf{Album}  с атрибутами: имя альбома (string), id альбома (integer), id персоны (integer), id фильма (integer). Однако при этом возникает несколько проблем:

\begin{itemize}
\item Поиск по идентификатору происходит намного быстрее, чем поиск на сравнение двух строк.
\item Хранить для каждого фильма отдельный альбом в виде строки не оптимально, это лишняя нагрузка на БД.
\end{itemize}

Избежать всего этого можно, выделив фотографии в отдельный класс. В результате мы получим:

\begin{figure}[h!]
\begin{center}
\includegraphics[scale=0.9]{image/class_my.png}
\end{center}
\caption{Диаграмма классов: индивидуальная часть}
\end{figure}

Из диаграммы видно, что теперь класс \textbf{Album} содержит id персон, фильмов и альбомов. Он связан отношением многие ко многим с классом \textbf{Image} (фотография), \textbf{Films} (фильмы) и \textbf{Persons} (персоны). Класс \textbf{Image} хранит информацию о фотографиях: К какому альбому принадлежит фотография(связь через id альбома), какой тип имеет(фоторафия или скриншот). Теперь, после определения классов и их атрибутов, которые будут в нашей системе, можно переходить к реализации.\\
Так как для каждого класса необходимы модель, контроллер и представления воспользуемся генератором \textit{scaffold}, и сгенерируем все части с его помощью.
\verbatiminput{code/sc.txt}
Внесем изменения в модель \textbf{Images:}
\verbatiminput{code/image_modt.txt}
Здесь указана связь многие ко многим с классом \textbf{Album} (belongs\_to \:album). Также добавлены ограничения: id альбома должен быть всегда не пустым. 
А также добавим изменения в главную модель: \textbf{Album:}. 

\verbatiminput{code/album_mod.txt}
Вначале указаны связи с классами \textbf{Image} (has\_many \:images), \textbf{Persons} (belongs\_to \:person), \textbf{Films} (belongs\_to \:film) и \textbf{User} (belngs\_to \:user). Также добавлены несколько проверок на корректность вводимых пользователем данных: имя альбома не может быть пустым, оно уникально и имеет длину от 3 до 40 символов. Также определена интересная валидация, используемая для того чтобы при создании альбома нельзя было привязать его сразу к персоне и фильму, либо сразу ни к персоне ни к фильму. Далее определены несколько методов, которые понадобятся в дальнейшем. Для удобного просмотра фотографий сделаем удобную навигацию с помощью плагина fancy_box, с возможнотью пролистывания фотографий или автоматического слайдшоу.\\

В моделях классов \textbf{Person} и \textbf{Films} также необходимо указать связи с классом \textbf{Album}.

\verbatiminput{code/p_f_mod.txt}

\subsection{Контроллер и Представление}
Теперь, после внесения всех необходимых изменений в модели, можно переходить к контроллерам. Контроллер интерпретирует данные, введённые пользователем, и информирует модель и представление о необходимости соответствующей реакции.\\
\hspace*{0.25cm}Для того чтобы пользователь мог добавлять альбомы, не нужно создавать отдельный интерфейс, можно сделать это при редактировании персоны или фильма. Для этого просто добавим ссылку на страничку создания альбома. 
Рассмотрим, как это будет выглядеть для класса \textbf{Album}.

\verbatiminput{code/form.txt}

Необходимо добавить небольшой скрипт, который будет посылать запрос в фоновом режиме к контролеру и ожидать результатов поиска. При этом контролер может извлекать данные из любого места, как, например, базы данных или жесткого диска. Но результаты поиска должны возвращаться  в формате \textit{JSON}.

\verbatiminput{code/js.txt}

Также необходимо добавить создание, отображение, редактирование, удаление, изменени и сохранение нового альбома в контроллере альбома.

\verbatiminput{code/update.txt} 

Вначале находится персона, у которой \textit{id} равен \textit{params[:id]}, этот параметр передается от формы при ее редактировании. После этого создаётся новый альбом и сохраняется. Если пользователь не ввел никакой информации, то альбом не сохраняется. Если не удалось сохранить альбом (а это возможно только в одном случае -- данные, введенные пользователем,  не прошли проверку), делается возврат в форму, и отображается сообщение об ошибке пользователю.\\
\hspace*{0.5cm}Для того чтобы сделать возможным просмотр всех альбомов у персоны или фильма, нужно внести изменения в представление персоны или фильма, а именно в метод 
\textit{show}.
\verbatiminput{code/show_person.txt}

Аналогичным образом реализуется просмотр альбома у фильма.
\verbatiminput{code/film_prize.txt}

\endinput


\newpage
\section{Графический интерфейс}
Приведем пример работы.\\
\begin{figure}[h!]
\begin{center}
\includegraphics[scale=0.35]{image/main.png}
\end{center}
\newpage
\caption{Домашняя страничка}
\end{figure}
\newpage
Далее рассматривается, как будет выглядеть интерфейс просмотра классов, курсов, дисциплин, групп, учеников и школ.\\
\begin{figure}
\begin{center}
\includegraphics[scale=0.25]{image/classes.png}
\caption{Таблица школьных классов слушателей ФДО}
\end{center}
\end{figure}

\begin{figure}
\begin{center}
\includegraphics[scale=0.25]{image/cources.png}
\caption{Таблица курсов слушателей ФДО}
\end{center}
\end{figure}

\begin{figure}
\begin{center}
\includegraphics[scale=0.25]{image/disciplines.png}
\caption{Таблица дисциплин слушателей ФДО}
\end{center}
\end{figure}

\begin{figure}
\begin{center}
\includegraphics[scale=0.25]{image/groups.png}
\caption{Таблица групп слушателей ФДО}
\end{center}
\end{figure}


\begin{figure}
\begin{center}
\includegraphics[scale=0.25]{image/schoolers.png}
\caption{Таблица учеников слушателей ФДО}
\end{center}
\end{figure}

\begin{figure}
\begin{center}
\includegraphics[scale=0.25]{image/schools.png}
\caption{Таблица школ}
\end{center}
\end{figure}

Приведем пример добавления нового учебного класса новый альбом, для этого достаточно перейти по ссылке <<Добавить>>. При нажатии на данную ссылку произойдет перенаправление на страничку создания учебного класса.\\

\begin{figure}
\begin{center}
\includegraphics[scale=0.25]{image/class_new.png}
\caption{Добавление учебного класса}
\end{center}
\end{figure}

Если просто нажать кнопку <<Сохранить>>, то отобразится сообщение об ошибке.\\

\begin{figure}
\begin{center}
\includegraphics[scale=0.25]{image/class_error.png}
\caption{Ошибка}
\end{center}
\end{figure}

В данном случае продемонстрирован наглядный пример работы валидатора: название класса должно существовать, т.е. не существует класса без названия и это название должно содержать не менее 1 символа в названии. А также учебный год обязательно должен быть числом.\\
В остальных классах реализован аналогичный интерфейс, вот иллюстрация его работы, на примере ученика.\\

\begin{figure}[ht]
\begin{center}
\includegraphics[scale=0.28]{image/schooler_edit.png}
\caption{Редактирование ученика}
\end{center}
\end{figure}

\begin{figure}[ht]
\begin{center}
\includegraphics[scale=0.28]{image/schooler_new.png}
\caption{Создание нового ученика}
\end{center}
\end{figure}

\begin{figure}[ht]
\begin{center}
\includegraphics[scale=0.35]{image/schooler_error.png}
\caption{Ошибки при сохранении ученика}
\end{center}
\end{figure}
\endinput

\newpage
\begin{thebibliography}{}

\bibitem{rlatex}
С.М. Львовский.
{\em Набор и вёрстка в системе \LaTeX, 3-е изд., испр. и доп.}~---
М., МЦНМО, 2003. Доступны исходные тексты этой книги.

\bibitem{wiki-LaTeX}
\link{http://ru.wikipedia.org/wiki/LaTeX}~---
Википедия (свободная энциклопедия) о системе \LaTeX.

\bibitem{sbras}
\link{http://www.sbras.ru/win/docs/TeX/LaTex2e/docs\_koi.html}~---
Различная документация по системе \LaTeX.

\bibitem{memoir}
\link{http://edgeguides.rubyonrails.org/}~---
Официальный сайт Ruby On Rails.

\bibitem{railscasts}
\link{http://railscasts.com/}
Видео уроки работы с Ruby On Rails.

\bibitem{apidoc}
\link{http://apidock.com/rails}
Электронная документация по Ruby On Rails 

\end{thebibliography}

\endinput

\end{document}

\endinput
